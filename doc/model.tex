\documentclass[11pt]{article}

\usepackage{fullpage}
\usepackage{hyperref}
\usepackage{amssymb}
\usepackage{amsmath}

\title{Alignment-free relative abundance with $K$-mers}
\author{Bob Carpenter}
\date{\today}

\usepackage[colorinlistoftodos,bordercolor=orange,backgroundcolor=orange!20,linecolor=orange,textsize=scriptsize]{todonotes}
\newcommand{\rob}[1]{\todo[inline]{\textbf{Robert: }#1}}
\newcommand{\guillaume}[1]{\todo[inline]{\textbf{Guillaume: }#1}}
\newcommand{\aaron}[1]{\todo[inline]{\textbf{Felix: }#1}}

\begin{document}

\maketitle

\abstract{\noindent Given RNA-seq data and a reference transcriptome,
  we provide a statistical model of relative abundance of transcripts
  without the need to align the reads to the transcriptome by reducing
  the reads and reference transcripts to $K$-mers.}

\rob{Maybe a diagram that schematizes the transform from RNA-seq to reads to k-mers into counts would help guide the introduction? It would also me!}
\section{Bases and $K$-mers}

The four \emph{DNA bases} are adenine (\texttt{A}),
cytosine (\texttt{C}), guanine (\texttt{G}), and thymine (\texttt{T}).
The set of DNA bases is
\[
  \mathbb{B} = \{ \texttt{A}, \texttt{C}, \texttt{G}, \texttt{T} \}.
\]
A \emph{$K$-mer} is an $K$-tuple of bases,
\[
  x = (x_1, \ldots, x_K),
  \ \textrm{where}
  \ x_1, \ldots, x_k \in \mathbb{B}.
\]
The set of $K$-mers is
\[
  \mathbb{B}^K
  = \{ (x_1, \ldots, x_K)
      : x_1, \ldots x_K \in \mathbb{B} \}.
\]
The set of $K$-mers of all lengths is
\[
  \mathbb{B}^* = \bigcup_{k=0}^{\infty} \mathbb{B}^k.
\]

\section{The transcriptome}

A \emph{transcriptome} is an indexed multiset of $K$-mers, each of
which may be of a different length,
\[
  T = T_{1}, \ldots, T_{G},
\]
with $T_{g, 1:K_g} \in \mathbb{B}^*$ for $g =1, \ldots, G$ and $K_g$ is the length of the $g$th transcriptome.  In data structure terms, the
transcriptome $T$ is a ragged array of bases.


\section{RNA-seq data}

Simple (unpaired) \emph{RNA-seq data}, after processing, consists of an indexed set of base
sequences, each of which is called a \emph{read},
\[
  R = R_1, \ldots, R_N, \ \textrm{where} \ R_n \in \mathbb{B}^J.
\]
Although not strictly necessary, we assume for simplicity that all of
the reads in the RNA-seq data are of the same length.

With \emph{paired-end} RNA-seq data, each read consists of two short RNA
sequences read from the same transcript with an unknown gap between
them on the transcriptome, so the data may be represented as
\[
  P = P_1, \ldots, P_N, \ \textrm{where} \ P_n \in \mathbb{B}^J \times
  \mathbb{B}^J.
\]
Again, although not strictly necessary, we assume all sequences are of
the same length to simplify notation.


\section{Shredding $K$-mers}

Suppose $K \leq N$.  Then the $N$-mer $x_{1:N}$ can be shredded into
a sequence of $(N - K + 1)$ $K$-mers by
\[
  x_{1:N} \ \mapsto \
  x_{1:K}, \ x_{2:K+1}, \ \ldots, \ x_{N-K+1:N}.
\]
For example, shredding   \texttt{AACAC}  into $2$-mers gives
\[
  \texttt{AACAC} \ \mapsto \
  \texttt{AA}, \ \texttt{AC}, \ \texttt{CA}, \ \texttt{AC}.
\]
A sequence (or multiset) of $K$-mers may be summarized with a function
$\textrm{count}_K:\mathbb{B}^* \rightarrow \mathbb{B}^K \rightarrow
\mathbb{N}$, where
$\textrm{count}_K(x)(y)$ is the number of times the $K$-mer
$y$ appears in the $N$-mer $x$.  Continuing the example above,
\begin{eqnarray*}
  \textrm{count}(\texttt{AACAC})(\texttt{AA}) & = & 1,
  \\
  \textrm{count}(\texttt{AACAC})(\texttt{AC}) & = & 2,
  \\
  \textrm{count}(\texttt{AACAC})(\texttt{CA}) & = & 1,
\end{eqnarray*}
and all other counts are zero.


A complete set of RNA-seq data $X = X_1, X_2, \ldots, X_N$ may then be summarized by its
$K$-mer counts by summing the counts of the individual reads $x_{n, 1:J}$
\[
  \textrm{count}_K(X) = \sum_{n=1}^N \textrm{count}_K(X_n).
\]

Paired-end RNA-seq data can be shredded elementwise.  For
example, consider extracting 3-mers from a paired read.
\[
  \texttt{ATCAG} / \texttt{CGCGC}
  \ \mapsto \
  \texttt{ATC}, \texttt{TCA}, \texttt{CAG},
  \texttt{CGC}, \texttt{GCG}, \texttt{CGC}.
\]
Counts are defined for individual paired reads and a complete set of
RNA-seq data as above.
% \rob{I don't understand this Paired-end RNA-seq part or why it makes sense to mix $k$-mers of the two pairs.}

\section{Indexing $K$-mers}

Define a bijection from the set of $K$-mers, $\mathbb{B}^K$, to the
first $4^K$ counting numbers, $\{ 0, 1, \ldots 4^K-1 \}$,
by treating each base as a digit in base 4 and reading the
$K$-mer as a number.  Following alphabetic order, let
\[
  \texttt{A} = 0,
  \quad \texttt{C} = 1,
  \quad \texttt{G} = 2,
  \quad \texttt{T} = 3.
\]
The resulting indexing sorts the $K$-mers into lexicographic order.
For example, with $K = 2$, the sixteen 2-mers are as follows.

\begin{center}
\begin{tabular}{ll|ll|ll|ll}
  \texttt{AA} & 0 & \texttt{CA} & 4 &   \texttt{GA} & 8 & \texttt{TA} & 12
  \\
  \texttt{AC} & 1 & \texttt{CC} & 5 &   \texttt{GC} & 9 & \texttt{TC} & 13
  \\
  \texttt{AG} & 2 & \texttt{CG} & 6 &   \texttt{GG} & 10 & \texttt{TG} & 14
  \\
  \texttt{AT} & 3 & \texttt{CT} & 7 &   \texttt{GT} & 11 & \texttt{TT} & 15
\end{tabular}
\end{center}
There are roughly one thousand 5-mers, because $4^5 = 1024$.  There are
roughly one million 10-mers and one billion 15-mers.

$K$-mer counts can be tabulated and stored very efficiently by using
this numbering to index into an array.  The complete set of 10-mer
counts from an RNA-seq experiment can be efficiently indexed and
stored in 4MB of memory using 32-bit counts (8MB with 64-bit counts).
Even smaller memory footprints may be achieved by packing smaller
integer representations; this could be beneficial if it enables the
counts to be stored in cache.  Shredding into 15-mers would take 4GB
(8GB) and clearly exceed cache capacity.  Beyond 15-mers, direct
indexing becomes prohibitive on readily available hardware, and either
a hashing scheme would be required or a lossy Bloom filter in order to
store approximate counts.  The memory used by hashing will be
proportional to the number of unique $K$-mers in the RNA-seq data.

Therefore,
$\textrm{count}_K(x)$ is a (sparse) $4^K$-vector where the
$K$-mers are indexed in lexicographic order (i.e., read as base 4
numbers).


\section{Statistical model}

In a transcriptome consisting of $G$ base sequences, the only
parameter is a vector $\alpha \in \mathbb{R}^G$, representing
intercepts in a multi-logit regression.  $\theta_g$ represents the
relative abundance of sequence $g$ in the transcriptome.
\rob{I don't understand this paragraph =D}
\subsection{Parameterizing simplexes}

A $K$-\emph{simplex} is a $K$-vector $\theta$ of non-negative values
such that $\textrm{sum}(\theta) = 1.$  An arbitrary $K$-vector $\alpha
\in \mathbb{R}^K$ can be transformed to a simplex using the softmax
function,
\[
  \textrm{softmax}(\alpha)
  = \frac{\exp(\alpha)}
         {\textrm{sum}(\exp(\alpha))},
\]
where $\exp(\alpha)$ is defined elementwise.   By construction,
$\textrm{softmax}(\alpha)$ is a simplex, because
$\textrm{softmax}(\alpha)_i > 0$ and
$\textrm{sum}(\textrm{softmax}(\alpha)) = 1$.

As a function from $K$-vectors to $K$-simplexes, softmax is many to
one, because adding a constant to each component of $\alpha$ leads to
the same result,

\[
  \textrm{softmax}(\alpha)
  = \textrm{softmax}(\alpha + \lambda \cdot \textbf{1}_K),
\]
where $\lambda \in \mathbb{R}$ and $\textbf{1}_K$ is $K$-vector
of ones. In other words,  softmax is  not an injective function.

\rob{We can however extend the softmax function into a one-to-one mapping by as follows}
The  softmax0 function defines a smooth, one-to-one mapping from
$\mathbb{R}^{K-1}$ to $K$-simplexes by
\[
  \textrm{softmax0}(\beta) = \textrm{softmax}([0 \ \, \beta]).
\]
That is, a zero element is appended to the front of the $(K-1)$-vector.
This changes the interpretation of the elements from being only
determined relative to each other to having their scale defined
relative to the fixed first element.

\subsection{Transcriptome matrix}

A sequence $T_g$ in the transcriptome can be transformed into a
$4^K$-simplex representing the probability of observing a given
$K$-mer given a read of transcript $g$.  The entire transcriptome can
then be converted to a $(4^K \times G)$-matrix $X$, where column $g$ is the
simplex representing the probability of a $K$-mer being observed given
that the read was from transcript $T_g$.

\subsubsection{Uniform read location}
\rob{At this point it's not clear to me why we are worried about the distribution of $K$-mers being uniform or not. Is it because we assume this at some point?}
Even if the reads are distributed uniformly along a transcript, the
distribution of $K$-mers will not be uniform because of edge effects
and because $K$-mers may appear more than once if $K$ is small.

For example, consider a transcript \texttt{ATGGCAATG} of nine bases.
If reads are size six and we select 4-mers within those reads,
the possibilities are as follows.
\begin{center}
  \begin{tabular}{l|l}
    \textit{Reads} & 4\textit{-mers}
    \\ \hline \hline
    \texttt{ATGGCA} & \texttt{ATGG, TGGC, GGCA}
    \\ \hline
    \texttt{\hspace{0.5em}TGGCAA} & \texttt{\hspace{3em}TGGC, GGCA, GCAA}
    \\ \hline
    \texttt{\hspace{1em}GGCAAT} & \texttt{\hspace{6em}GGCA, GCAA, CAAT}
    \\ \hline
    \texttt{\hspace{1.5em}GCAATG} & \texttt{\hspace{9em}GCAA, CAAT, AATG}
  \end{tabular}
\end{center}
The distribution of 4-mers resulting from uniformly distributed reads
is not expected to be uniform, but rather as follows.
\begin{center}
  \begin{tabular}{r|c||r|c}
    3\textit{-mer} & \textit{Proportion} & 3\textit{-mer} & \textit{Proportion}
    \\ \hline \hline
    \texttt{ATGG} & 1/12 & \texttt{GCAA} & 3/12
    \\ \hline
    \texttt{TGGC} & 2/12 & \texttt{CAAT} & 2/12
    \\ \hline
    \texttt{GGCA} & 3/12 & \texttt{AATG} & 1/12
  \end{tabular}
\end{center}

Assuming each position in the transcript is equally likely to produce
a read of size $J$, then $X_{1:4^K, g}$ is the $4^K$-simplex defined
by composing count functions,
\[
  X_{1:4^K, \, g}
  = \frac{\textrm{count}_K\left(\textrm{count}_J(T_g)\right)}
         {\textrm{sum}\left(\textrm{count}_K\left(\textrm{count}_J(T_g)\right)\right)},
\]
where the outer function $\textrm{count}_K$ is type lifted additively
over functions as in the example.  Specifically, for a sequence $s$,
and integers $K \leq J \leq \textrm{size}(s)$, the $4^K$-vector of
$K$-mer counts found in uniformly distributed reads of size $J$ is
\[
  \textrm{count}_K(\textrm{count}_J(s))
  = \sum_{j = 0}^{4^J - 1} \textrm{count}_J(s)(j) \cdot \textrm{count}_K(j).
\]
In other words, the result $\textrm{count}_K(\textrm{count}_J(s))$ is a
function from $K$-mer identifiers to their relative abundance in
uniformly distributed $J$-gram reads over the sequence $s$.  The term
$\textrm{count}_K(j)$ is a sparse vector mapping $K$-mers to their
count in the $J$-mer represented by the integer $j$.  This is then
weighted by $\textrm{count}_J(s)(j)$, which is the count of the $J$-mer $j$ in
the sequence $s$.

\subsubsection{Non-uniform read location}

The probabilty of reads is non-uniform for several reasons.  Two
causes that have large impacts are hexamer (6-mer) binding and
position within the transcript.  In the first case, the probability of
reads being chosen from a transcript varies up to two orders of
mangitude or more based on the hexamer appearing in the first six
positions of the read.  Furthermore, the strength of these effects
depends on the priming protocol used.  In the second case, the
probability of a read can vary up to an order of magnitude or more
based on its relative position along the transcript.

Given the relative probabilities of reads originating at 6-mers, the
uniform transcriptome matrix may be adjusted by reweighting based on
the probability of the 6-mers.

Given a formula for relative probability along the transcript, the
transcriptome matrix may be further adjusted for positional effects.
In the end, the result is still a $(4^K \times G)$ transcriptome
matrix $X$.

Taken together, suppose we have transcript $T_g$ of size $N$, reads
of size $J$, and $p(n)$ is the probability of a read of size $J$
originating from position $n$.



\subsection{Observed data}

The observation $y$ is a sparse $4^K$-vector of $K$-mer counts
corresponding to the shredded reads.

\subsection{Transformed transcriptome matrix}

The transcriptome is representing as a $4^K \times G$ matrix $X$ as
described above.  The parameter $\alpha \in \mathbb{R}^G$ is a
$G$-vector whose values indicate the relative level of expression of
each transcript on the log scale; $\theta = \textrm{softmax}(\alpha)$
is the corresponding $G$-simplex representing the relative abundance
$\theta_g$ of each transcript $T_g$ in the transcriptome.

The key step is the transform of the transcript abundances,
represented by $\theta = \textrm{softmax}(\alpha)$, to $K$-mer
abundances, as represented by
$\phi = X \cdot \textrm{softmax}(\alpha)$.  Because
$\textrm{softmax}(\alpha)$ is a $G$-simplex and the columns of $X$ are
$4^K$-simplexes, the product $X \cdot \textrm{softmax}(\alpha)$ is a
$4^K$ simplex---it just reweights the gene-level $K$-mer expressions
represented by the columns by the gene expressions represented by
$\textrm{softmax}(\alpha)$.

\subsection{Sampling distribution}

The sampling distribution $p(y \mid X, \alpha)$  for the observed
$K$-mer counts $y$ given the transcriptome matrix $X$ and log odds
expression simplex $\theta$ is defined to be multinomial,
\[
  y \sim \textrm{multinomial}(X \cdot \textrm{softmax}(\alpha)).
\]


\subsection{Prior}

The prior is a normal centered at the origin,
\[
  \alpha_g \sim \textrm{normal}(0, \lambda),
\]
for some scale $\lambda > 0$.  The smaller the scale $\lambda$, the
more parameter estimates for $\alpha_g$ are shrunk toward zero, and
thus the less relative difference in expression is allowed.

If we intend to get true zero estimates for some genes with a
continuous prior, we will need to make penalized maximum likelihood
estimates with an L1-type penalty.  Keeping to sampling notation, this
is represented by a double-exponential exponential (aka Laplace)
distribution,
\[
  \alpha_g \sim \textrm{double-exponential}(0, \lambda).
\]

\subsection{Posterior}

The posterior is determined by Bayes's rule up to an additive constant
that does not depend on the parameters $\alpha$ as
\[
  \log p(\alpha \mid X, y) = \log p(y \mid X, \alpha) + \log p(\alpha) +
  \textrm{const.}
  \]

\subsubsection{Posterior gradient}

Efficient maximum penalized likelihood estimation and full Bayesian inference
require evaluating gradients of the posterior with respect to the parameters
$\alpha$.  The gradient is\footnote{The derivative is easy to work out
  by passing \url{matrixcalculus.org} the query
  \begin{center}\texttt{y' * log
(X * exp(a) / sum(exp(a))) - a' * a / (2 * lambda)}\end{center}
and then simplifying using $\textrm{softmax}$.}
%
\[
  \begin{array}{l}
  \nabla\!_{\alpha} \, \log p(\alpha \mid X, y) + \log p(\alpha)
  \\[4pt]
  \qquad = \
    t \odot \textrm{softmax}(\alpha)
    - \textrm{softmax}(\alpha)^{\top}\! \cdot t \cdot \textrm{softmax}(\alpha)
    - \frac{\displaystyle \alpha}{\displaystyle \lambda},
\end{array}
\]
where
\[
  t = X^{\top}\! \cdot (y \oslash (X \cdot \textrm{softmax}(\alpha)))
\]
and $\odot$ represents elementwise multiplication and $\oslash$
elementwise division.

\section{Estimating expression}

\subsection{Bayesian posterior mean estimate}

The Bayesian \emph{posterior mean estimate} is the parameter's
expected value conditioned on the observed data,
\begin{eqnarray*}
  \widehat{\alpha}
  & = & \mathbb{E}\!\left[\alpha \mid X, y \, \right]
  \\[6pt]
  & = & \int_{\mathbb{R}^G} \alpha \cdot p(\alpha \mid X, y) \, \textrm{d}\alpha.
\end{eqnarray*}
The posterior mean estimate minimizes expected square estimation error
for $\alpha$ if the model is well specified.

Given a sampler that can produce Markov chain Monte Carlo (MCMC) draws
distributed according to the posterior
\[
  \alpha^{(m)} \sim p(\alpha \mid X, y),
\]
the posterior mean estimate may be calculated as
\[
  \widehat{\alpha} \approx \frac{1}{M} \sum_{m = 1}^M \alpha^{(m)}.
\]
As $M \rightarrow \infty$, the approximation converges to the true
value.  Assuming the Markov chain is geometrically ergodic, the MCMC
central limit theorem applies, providing a convergence rate of
$\mathcal{O}\left(1 / \sqrt{M}\right)$.  In practice, a few hundred
effective draws brings the expected error on $\widehat{\alpha}$ due to
MCMC down to less than a tenth of the posterior standard deviation of
$\alpha$.

\subsection{Maximum a posteriori estimate}

The \emph{max a posteriori} (MAP) estimate for $\alpha$ is given by
choosing the parameter value $\alpha^*$ that maximizes the posterior
density,
\[
  \alpha^* = \textrm{arg\,max}_{\alpha} \,
  \log \textrm{multinomial}(y \mid x \cdot \textrm{softmax}(\alpha))
  + \log \textrm{normal}(\alpha \mid 0, \lambda \cdot \textrm{I}).
\]
This is not a Bayesian estimate because it does not average over
posterior uncertainty.


\subsection{Maximum penalized likelihood estimate}

A pure frequentist estimate may be defined by casting the normal prior
on the parameters as a penalty function based on the $\textrm{L}_2$
norm,
\[
  -\frac{1}{2 \cdot \lambda^2} \cdot \alpha^{\top}\!\! \cdot \alpha.
\]

Replacing the normal prior in the definition of the MAP estimate with
the $\textrm{L}_2$ penalty defined above, the
\emph{penalized maximum likelihood estimate} (MLE) is defined as
\[
  \alpha^* = \textrm{arg\,max}_{\alpha} \,
  \log \textrm{multinomial}(y \mid x \cdot \textrm{softmax}(\alpha))
  - \frac{1}{2 \cdot \lambda^2} \cdot \alpha^{\top}\!\! \cdot \alpha.
\]


\section*{References}

\begin{itemize}
\item Biases in Illumina transcriptome sequencing caused by random
  hexamer priming Kasper D. Hansen,1,* Steven E. Brenner,2 and
  Sandrine Dudoit1,3.  Nucleic acids research.
\end{itemize}

\end{document}
